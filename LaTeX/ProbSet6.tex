\documentclass[11pt]{article}
\usepackage{mathtools}
\usepackage{amsmath}

\title{\textbf{Homework}}
\author{Brian Chuk}
\date{3/4/2015}
\begin{document}
\maketitle

\section{Problem Set 6}
\begin{enumerate}
\item procedure secondmax($a_1 , a_2 , . . . , a_n$ : integers)
	\\max := $a_1$
	\\for i := 2 to n
	\\ \hspace*{20 pt}if $max <a_i$, then max := $a_i$
	secondmax := $a_1$
	\\for i := 2 to n
	\\ \hspace*{20 pt}if $secondmax < a_i$ and $a_i < max$, then secondmax := $a_i$
	\\return secondmax\{secondmax is the second largest element\}
\item procedure lasteven($a_1 , a_2 , . . . , a_n$ : integers)
	\\result := 0
	\\for i := 1 to n
	\\ \hspace*{20 pt}if $a_i \% 2 = 0$
	\\ \hspace*{40 pt}result := i
	\\return result\{result is the location of the last even integer (0 if there are no even integers)\}
\item procedure highlowaverage($a_1 , a_2 , . . . , a_n$ : integers)
	\\high := $a_1$
	\\low := $a_1$
	\\for i := 2 to n
	\\\hspace*{20 pt}if $high <a_i$, then high := $a_i$
	\\\hspace*{20 pt}if $low <a_i$, then low := $a_i$
	\\return (high + low)/2{the average of the largest and smallest integers in the list}\\\\\\
\item procedure maxproduct($a_1 , a_2 , . . . , a_n$ : integers)
	\\maxproduct := $a_1 * a_2$
	\\for i := 1 to n
	\\\hspace*{20pt}for j := i + 1 to n
	\\\hspace*{20pt}if maxproduct < $a_i * a_j$, then maxproduct = $a_i * a_j$
	\\return maxproduct\{the largest product between 2 numbers in a list\}
\item The algorithm will look at the center element of the list, which is 12. Since $27 > 12$, it will observe the latter half of the list. The binary search algorithm will then split the list in half again and compare 27 to the center element, which is 21. Since $27 > 21$, it will look at the latter half. Since $27 > 25$ but $27 < 31$, the algorithm will look in between 25 and 31 but not find 27.
\item The greedy algorithm first chooses a 12-foot-long board, and then three one-footlong
boards. This requires four boards. But only three boards are needed: each five
feet long.
\item This is true. Since the each coin divides the face value of every larger coin, a single larger coin will always represent an integer multiple of smaller coins.
\item This is false. 24 cents can be made with 3 8-cent coins but the greedy algorithm will use 1 20-cent count and 4 1-cent coins.
\item $1^2 + 2^2 +...+n^2 \leq n^2 + n^2 + ... + n^2 = n * n^2 = n^3$
\item $\frac{3n-8-4n^3}{2n-1} \leq \frac{3n^3+8n^3+4n^3}{2n-n}=\frac{15n^3}{n}=15n^2$
\item $1^3 + 2^3 +...+n^3 \leq n^3 + n^3 + ... + n^3 = n * n^3 = n^4$
\item $\frac{6n + 4n^5 - 4}{7n^2 - 3} \leq \frac{6n^5 + 4n^5 + 4n^5}{7n^2-3n^2} = \frac{14n^5}{4n^3} = \frac{7}{2}n^3$
\item $1 * 2 + 2 * 3 + 3 * 4 + ... + (n-1) * n \leq 1 * 2n + 2 * 3n + 3 * 4n +...+(n-1)*n^2 \leq n^3$ 
\item $n^2$
\item $n^2$
\item $n$
\item $n^4$
\item $n^4$
\item $\sum\limits_{j=1}^n (j^3 + j) = (1^3 + 1) + (2^3 + 2) + ... (n^3 + n) \leq (n^3 + 1) + (n^3 + 2) + ... (n^3 + n)= n(n^3 + n) = n^4 + n^2$
\item $log_2(x^2+1)$ and $log_2(x^3+1)$ are both $O(log_2x)$. Each term is thus $O(xlog_2x)$ and hence so is the sum.
\item $\frac{x^3+7x^2+3}{2x+1} \leq \frac{x^3+7x^3+3x^3}{2x} = \frac{11x^3}{2x} = \frac{11}{2}x^2$
\item 
\item 
\item 
\item 
\item 
\item 
\item 
\item 
\item 
\item 
\item 
\item 
\item 

\end{enumerate}
\end{document}