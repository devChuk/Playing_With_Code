\documentclass[11pt]{article}
\usepackage{mathtools}
\usepackage{amsmath}

\title{\textbf{Homework}}
\author{Brian Chuk}
\date{2/11/2015}
\begin{document}
\maketitle

\section{Problem Set 3}
\begin{enumerate}

\item
	a)	$P(1) = 1 * 1! = (1+1)! - 1 = 1$\\
	b)	$P(5) = 5 * 5! + 4 * 4! + 3 * 3! + 2 * 2! + 1 * 1! = 600 + 96 + 18 + 4 = (5+1)! - 1 = 718$\\
	c)	$P(k) = k * k! + (k-1)*(k-1)!+...1*1!$\\
	d)	$P(k+1) = (k+1) * (k+1)! + k * k! + (k-1)*(k-1)!+...1*1!$\\
	e)	Assuming that for an arbitrary integer $k$, \\$P(k) = k * k! + (k-1)*(k-1)!+...1*1! = (k+1)!-1$ \\
	Therefore, $P(k+1) = (k+2)!-1$ is true.\\
	When we add $(k+1) * (k+1)!$ to both sides of the equation $P(k)$, we obtain\\
	$1 * 1! + ... + (k-1)*(k-1)! + k * k + (k + 1) * (k+1)!$\\
	$ = (k+1)!-1 + (k + 1) * (k + 1 )! $ \\
	$ = (k+1)k! + (k + 1)(k + 1)! - 1$ \\
	$ = (k+1)(k!(1 + (k + 1))) - 1$ \\
	$ = (k+1)(k!(k + 2)) - 1$ \\
	$ = k! * (k+1) * (k+2) - 1$ \\
	$= (k + 2)! - 1$
\item
	$P(2)$ is true, because $8 > 7$\\
	Assume for an integer $k$ greater than or equal to 2, $k^3 > k^2 + 3$\\
	For $k+1$, $(k+1)^3 > (k+1)^2 + 3$\\
	$(k + 1)^2 + 3$
	$= k^2 + 2k + 1 + 3$\\
	$= (k^2 + 3) + 2k + 1 < k^3 + 2k + 1 \leq k^3 + 3k \leq k^3 + 3k^2 + 3k + 1 = (k + 1)3$
\item
	Let $P(n) = 1 + 3 + 9 + ... + 3^n$ for all $n > 0$.\\
	$P(1)$ is true because $3^1 = (3^{n+1} - 1)/2= 4$\\
	$1 + 3 + ... + (3^{k-1} - 1)/2 + 3^{k+1} = \dfrac{3^{k+1} - 1 + 2 * 3^{k+1}}{2}$\\
	$\dfrac{3^{k+2}-1}{2}$
\item
	Let $P(k)$ represent the stamps needed to satisfy a postage of k cents.\\
	$P(18)$ is true because it uses one 4-cent stamp and two 7-cent stamps.\\
	If $P(k)$ has a 7-cent stamp, replace one 7-cent stamp with two 4-cent stamps. If the pile contains only 4-cent stamps and there are over five 4-cent stamps, replace five 4-cent stamps with three 7-cent stamps.
\item
	$\sum\limits_{j=1}^1 (2j+1) = 3 = 3*1^2$, so the base case is true.\\
	Assume $\sum\limits_{j=k}^{2k-1} (2j+1) = 3k^2$. Therefore,\\
	$\sum\limits_{j=k+1}^{2(k+1)-1} (2j+1) = \sum\limits_{j=k}^{2k-1} (2j+1) - (2k +1) + (4k + 1) + (4k+3)$\\
	$= 3k^2+ 6x + 3$\\
	$=3(k+1^2)$
\item
	Since one line in the plane passing through one point divide the plane into $2(1)$ regions, the basis step holds.\\
	Assume $k$ lines in a plane pass through the same point and divide the plane into 2k regions. Adding another line splits exactly two of these regions into two parts each. Therefore, $k+1$ lines that meet at the same point split the plane into $2k+2 = 2(k+1)$ regions.
\item
	$a_1 < 3^1$ is true because $2 < 3$. The base step is true.\\
	Let $P(n)$ be the proposition $a_n < 3^n$.Assume $P(n)$ is true for all $n$ when $1 \leq k < n$.\\
	So ,$a_n = 2a_{n - 1} + 3a_{n - 2} \leq 2 * 3^{n − 1} + 3 * 3^{n − 2} $\\
	$= 2 * 3^{n - 1} + 3^{n - 1} $\\
	$= 3 ⋅ 3^{n - 1}$\\
	$= 3n$
\item
	There is no basis case.
\item
	$f (n) = f (n - 1) + 5$
\item
	$a_n = \sqrt{a_{n-1}} $
\item
	Assume $p$ is true and $i \leq n$ is true at the end of a rotation of the loop. If the loop is executed again, $i$ would be incremented by 1 and $i \leq n$. Total would become $\dfrac{(i-1)i}{2} + i = \dfrac{(i+1)i}{2}$. Therefore, $p$ is true.
\end{enumerate}
\end{document}