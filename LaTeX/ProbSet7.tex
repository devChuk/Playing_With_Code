\documentclass[11pt]{article}
\usepackage{mathtools}
\usepackage{amsmath}

\title{\textbf{Homework}}
\author{Brian Chuk}
\date{3/4/2015}
\begin{document}
\maketitle

\section{Problem Set 6}
\begin{enumerate}
\item False: $2|8, 3|6$, but $(2*3)\not|14$ because $14/6$ is not an integer.
\item True: Since $a|(b + c)$, then there exists $x$ and $y$ such that: \\$(b+c) = a(x+y)$.\\ Thus, $b = ax$ and $c = ay$.\\Therefore, $a|b$ and $a|c$.
\item True: Since $a|c$ and $b|c$, then there exists $x$ and $y$ such that: \\$c=ax$, $c=by$\\ By multiplying the equations together, we obtain $c^2=ab(xy)$. \\Therefore, $ab|c^2$.
\item 19, 17, 13, 11, 9, 7, 3, 1
\item $12!$
\item $2^346$
\item $0$
\item False: $(2 + 6\sqrt7) + (-6\sqrt7) = 2$
\item False: $2^{\frac{1}{2}} = \sqrt2$
\item True: $a \equiv b($mod $ m) $. Therefore, there exists integer $c$ such that $m = (a-b)c$.\\
If the equation is multiplied by 2, we obtain: $2m = (2a-2b)c$.\\
As a result, $2a \equiv 2b($mod $ 2m) $
\item False: $a \equiv b($mod $ m^2) $, then there exists integer $c$ such that $m^2 = (a-b)c$.\\
$a \equiv b($mod $ m) $ cannot be true because $\sqrt{c}$ may not be an integer as well.
\item $a|b$ represents a relation between two integers, while $\frac{b}{a}$ is an expression.
\item 101111000001
\item $k$ in $b = ak$ and $c = bk$ may not be the same $k$.
\item True: $n$ is not a multiple of 3, so there are two cases for $n$.\\
Case 1:If $n = 3k + 1$, then $n^2 = 9k^2 + 6k + 1 = 3(3k^2 + 2k) + 1$. \\Therefore, $n \equiv 1($mod $ 3) $.\\
Case 2:If $n = 3k + 2$, then $n^2 = 9k^2 + 12k + 4 = 3(3k^2 + 4k + 1) + 1$. \\Therefore, $n \equiv 1($mod $ 3) $.
\item 1
\item 0 1100
\item 1 0011
\item $t + 11k$
\item 
\item 15
\item 
\item 
\item 
\item 
\item 
\item 
\item 
\item 
\item 
\item 
\item 

\end{enumerate}
\end{document}