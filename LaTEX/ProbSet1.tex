\documentclass[11pt]{article}
\usepackage{mathtools}

\title{\textbf{Homework}}
\author{Brian Chuk}
\date{2/4/2015}
\begin{document}
\maketitle

\section{Problem Set 1}
\begin{enumerate}

\item $ p \wedge r \wedge \lnot q $
\item True; they are identical
\item It is not a tautology.
\item $ w \rightarrow  c $
\item It is satisfiable. True when p is false and q is true.
\item false 
\item false
\item false
\item false
\item false
\item Not valid
\item Let R(x) be the predicate "x has read the textbook," and P(x) be the predicate "x passed the exam"\\
$ \forall x (R(x) \rightarrow P(x)) \\ R(Ed) \rightarrow P(Ed) \\ R(Ed) \\ P(Ed)$ 
\item The propositions do not imply the conclusion
\item Existential generalization
\item Let $n = 2a + 1$ and $n^2 = 2b$. So, $4a^2 + 4a + 1 = 2b$. Therefore, $4a^2 + 4a - 2b = -1$. This is a contradiction because there is an even = odd. Thus, $n^2$ is odd.
\item Let $x = 2a + 1$. Thus $x + 2 = 2a + 3$, which is odd. 
\item Let $x + 2 = 2k$. Therefore $x = 2k - 2 = 2(k-1)$, which is even.
\item Suppose $x$ is odd but $x + 2$ is even. Thus $x = 2a + 1$ and $x + 2 = 2l$. Therefore $x + 2 = 2a + 3 = 2l$. Since $2a - 2l = -3$, there is a contradiction because even = odd.
\item Let $m = 2a$ and $n = 2b$. Therefore $mn = 4ab$, which is a multiple of $4$.
\item False when $x = -2$ and $y = 0.5$
\item Let $n = 2a$ and $n^3 + 1 = 2b + 1$. Therefore $n^3 + 1= 8a^3 + 1 = 2b + 1$. $8a^3 - 2b = 0$. Since it is even = even, if $n$ is even then $n^3 + 1 $ is odd. \\
Now, since $n^3 + 1 = 2b + 1$ then $n = \sqrt[3]{2b} = 2a$. The cube root of an even number is always even, therefore this is an even = even equation. Thus, if $n^3 + 1$ is odd then $n$ is even. Thus, the statements $n$ is even and $n^3 + 1$ is odd are equivalent.

Let $n = 2a$ and $n^3 - 1 = 2b + 1$. Therefore $n^3 + 1 = 8a^3 + 1 = 2b + 1$. As a result, $8a^3 - 2b = 0$. Since this is even = even, if $n$ is even then $n^3 - 1$ is odd.
Now, since $n^3 - 1 = 2b + 1$ then $n = \sqrt[3]{2b + 2} = 2a$. $n = \sqrt[3]{2(b+1)}$ is an even = even equation, therefore if $n^3 - 1$ is odd then $n$ is even.

Thus the statements $n$ is even and $n^3 -1$ are equivalent. In conclusion, the three statements are equivalent.

\item If three people were born in each of the months of the year at most, there would be 36 people. Thus, those 4 remaining people must have birthdays in months were at least three others have birthdays. In conclusion, at least four people were born in the same month of the year out of any 40 people.
\item We can reduce the possible values of $x$ and $y$ into a few cases because $2x^2 > 14$ when $|x|\geq 3$ and $y^2> 14$ when $|y| \geq 4$. This leaves the cases when $x$ equals $-2,-1,0,1,$ or $2$ and $y$ equals $-3, -2, -1, 0, 1, 2,$ or $3$. Since we are only testing for positive integer solutions we can remove the negative integers from the cases we have to test. As a result, none of the six remaining cases are solutions to the equation. In conclusion, the equation $2x^2+y^2=14$ has no integer solutions.
\end{enumerate}


\section{Problem Set 2}
\begin{enumerate}
\item $\overline{A \bigcap B} \\= \lbrace x | x \not\in A \bigcap B\rbrace$
\\$= \lbrace x | \lnot(x \in A \bigcap B)\rbrace$
\\$= \lbrace x | \lnot(x \in A \bigwedge x \in B)\rbrace$
\\$= \lbrace x | \lnot(x \not\in A \bigvee x \not\in B)\rbrace$
\\$= \lbrace x | \lnot(x \in \overline{A} \bigvee x \in \overline{B})\rbrace$
\\$= \lbrace x | \lnot(x \in \overline{A} \bigcup \overline{B})\rbrace$
\\$\overline{A} \bigcup \overline{B}$
\item true
\item true
\item false
\item 1
\item false
\item true
\item 0
\item uncountable
\item Both (0,1) and (0,2) are uncountably infinite so they have the same cardinality.
\item f is 1-1
\item yes
\item f is not 1-1
\item yes
\item f is not 1-1
\item yes
\item g is 1-1 and onto R
\item \{2,3,8\}
\item \{(8,1),(3,2),(2,3)\}
\item yes
\item no
\item $a_n = 3^n + 1 = a_n = 3^n + 1^n$\\
$(s-3)(s-1) = 0$\\
$s-4s+3=0$\\
$s^2-As-B = 0$\\
$A=4,B=-3$\\
$a_n = Aa_{n-1} + Ba_{n-2}$\\
$a_n = 4a_{n-1} - 3a_{n-2}$\\
\item $a)$ $25000(1 + 0.03)^n$\\
$b)$ $25000(1 + 0.05)^n$\\
$c)$ $(25000+1000n)(1 + 0.02)^n$
\end{enumerate}
\end{document}
